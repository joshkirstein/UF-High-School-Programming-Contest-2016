\documentclass[a4paper,11pt]{article}
\usepackage[left=0.85in,right=0.85in,top=0.80in,bottom=0.80in]{geometry}
\usepackage{listings}
\usepackage{amsmath}
\usepackage{fmtcount}
\usepackage{datetime}
\usepackage[pdftex]{graphicx}
\usepackage{fancyhdr}
\usepackage{color}
\usepackage{fancyvrb}
\usepackage{tabu}
\usepackage{xcolor}
\begin{document}
\begin{center}
{\Huge Problem ID: multiply}\vspace{2 mm} \\	% Problem Letter
{\huge RSA Cryptosystem}\vspace{2 mm} \\	% Problem Name

\setcounter{page}{12}
\end{center}
\large{
Your friends Alice and Bob are very secretive people. Whenever they send a message to each other they encrypt it using the \emph{RSA} algorithm. For the algorithm to work, Alice and Bob must each pick two very large prime numbers $p$ and $q$ and from these two numbers they can follow the RSA procedure to generate their own public-key and private-key.
\\\\
To send an encrypted message to Alice, Bob would have to take her public-key and encrypt his message with it; the only way to decrypt this message is with the private-key that Alice has. One part of the public-key that Alice releases is the cryptographic modulus: $n = p*q$. Alice and Bob have decided to enlist your efforts to help them compute this cryptographic modulus. 
\\\\
Alice and Bob will provide two numbers, $p$ and $q$ ($1 \leq p$, $q$ $\leq 10^{1000}$). Fortunately, they aren't very good at math and don't really know what prime numbers are; instead they will provide you numbers of a very specific form. Both numbers will be a single non-zero digit followed by zeros (it's possible for there to be no zeros following the non-zero digit, but there will always be a non-zero digit that starts).}
\vspace{7mm}\\
\large{\bf{Input}}\vspace{2mm}\\
The input will begin with a line containing a single positive integer $t$ representing the number of modulus values you must compute. Following will be $t$ lines each containing two space-separated integers $p$ and $q$ ($1 \leq p$, $q$ $\leq 10^{1000}$).
\vspace{3mm}\\
\large{\bf{Output}}\vspace{2mm}\\
For each test case, print the modulus ($p*q$) on its own line.
\vspace{5mm}\\
\bf{Sample Input} \hspace{44mm} \bf{Sample Output}\vspace{1mm}\\
\begin{tabu*} to 425pt {|X[0r]|X[0l]|}
\tabucline-
\vspace{-\baselineskip} %needs to be placed here
\begin{Verbatim}
4
5 9
700000 700000
10 1
1000000 9
\end{Verbatim}
&
\vspace{-\baselineskip} %needs to be placed here
\begin{Verbatim}
45
490000000000
10
9000000
\end{Verbatim}
\\
\tabucline-
\end{tabu*}
\end{document}