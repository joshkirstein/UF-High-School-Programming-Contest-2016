\documentclass[a4paper,11pt]{article}
\usepackage[left=0.85in,right=0.85in,top=0.80in,bottom=0.80in]{geometry}
\usepackage{listings}
\usepackage{amsmath}
\usepackage{fmtcount}
\usepackage{datetime}
\usepackage[pdftex]{graphicx}
\usepackage{caption}
\usepackage{subcaption}
\usepackage{fancyhdr}
\usepackage{color}
\usepackage{fancyvrb}

\usepackage{tabu}
\usepackage{xcolor}\newcommand{\squeezeup}{\vspace{-7mm}}
\begin{document}

\setcounter{page}{4}
\begin{center}
{\Huge Problem ID: dna}\vspace{2 mm} \\	% Problem Letter
{\huge DNA Replication}\vspace{2 mm} \\	% Problem Name
\end{center}
\large{
Deoxyribonucleic acid (also known as \emph{DNA}) is a molecule that carries most of the genetic instructions used to power life on Earth. At a high level, DNA is composed of the fundamental primitives adenine (A), thymine (T), guanine (G), and cytosine (C) which are collectively known as nucleotides. Nucleotides are essentially the 1s and 0s of life (except DNA is more like base-4 instead of base-2). \emph{DNA sequencing} is the process of determining the precise order of nucleotides within a given DNA molecule; the output of this process is a string containing `A', `T', `G', or `C'.\\\\
A \emph{repeated substring} is a substring that has at least two occurrences in the given string (occurrences can overlap). Repeated substrings of a DNA sequence are important because they often indicate the start/stop of an important sequence. If a repeated substring is the longest out of all of them we call it a \emph{longest repeated substring}. It is possible for there to be multiple longest repeated substrings. \\\\
Given a string corresponding to a sequenced DNA molecule, can you determine the length of a longest repeated substring as well as the number of distinct repeated substrings that have this length?
}
\vspace{7mm}\\
\large{\bf{Input}}\vspace{2mm}\\
The input will begin with a line containing a single positive integer, $t$, representing the number of test cases to process. Each test case will consist of a single line containing the characters `A', `T', `G', or `C'. The input string is guaranteed to be at most 50 characters long.
\vspace{3mm}\\
\large{\bf{Output}}\vspace{2mm}\\
For each test case print two space-separated integers on their own line: the length of a longest repeated substring and the number of distinct repeated substrings that have this length. Output ``0 0" if there is no repeated substring.
\vspace{10mm}\\
\bf{Sample Input} \hspace{52mm} \bf{Sample Output}\vspace{1mm}\\
\begin{tabu*} to 475pt {|X[0r]|X[0l]|}
\tabucline-
\vspace{-\baselineskip} %needs to be placed here
\begin{Verbatim}
6
ATATA
ATATAT
CCCCC
ACTGGGACT
ACTG
CCCAGGG
\end{Verbatim}
&
\vspace{-\baselineskip} %needs to be placed here
\begin{Verbatim}
3 1
4 1
4 1
3 1
0 0
2 2
\end{Verbatim}
\\
\tabucline-
\end{tabu*}
\end{document}
